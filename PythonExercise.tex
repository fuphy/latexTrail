\documentclass{article}
    \title{Learning Python with your friends}
    \author{Ankur}
    \begin{document}
    \maketitle{}
    \section*{Excercise 1}
    \subsection*{Task 1}
    Install Anaconda and an editor (Visual Studio Code or Spyder preferably)
    \subsection*{Task 2}
    Print "Hello, World!"
    \subsection*{Task 3}
    Try to understand what is 'variable' and 'data type' and what are the different types of data type in Python
    %This is where you will write your content.
    \subsection*{Task 4}
    Declare variables with relevant data type and print:
    \paragraph{Name}
    \paragraph{Age}
    \paragraph{List of your 5 favorite countries}
    \paragraph{Average number of hours you work/study every day with 1 decimal value}
    \paragraph{Dictionary of your name, place you study/work, degree/position}
    \\
    \section*{Excercise 2}
    \subsection*{Task 1} For and while loop
    \paragraph{1.1}
    Print 1 to 10 using for and while loop
    \paragraph{1.2}
    Print 5 to -5 using for and while loop
    \subsection*{Task 2} Create a list of -10 to 10 and repeat Task 1 using for loop
    \subsection*{Task 3} Use the list created in Task 2 and print all the elements in reverse order
    \subsection*{Task 4} Create a dictionary of 5 items and print the key and value using for loop
    \subsection*{Task 5} 
    Write a program where you have a variable denoting temperature of your home town in celsius and print the output in Fahrenheit upto two decimals
    \\
    \section*{Excercise 3}
    For last 3 questions, you need to write your own function. Do not use any inbuilt function or any library
    \subsection*{Task 1} Print Fibonacci series up to 100
    \subsection*{Task 2} Print the factorial of a number
    \subsection*{Task 3} Define a function to return if numbers from 1 to 10 are odd or even
    \subsection*{Task 4} Define a function to check if a given string is a palidrome
    \subsection*{Task 5} Define a function to check if numbers between 1-20 are prime or not
    
    \end{document}